
% TTD
% 1. Table for SI 2. Add regions on contour 3. finish the Li-air part 4. emissions 5. Common terms like narrow-body/ narrow body etc need to be fixed



%In an earlier work, we analyzed the performance limitations hindering mass-market electric light commercial vehicles and identified battery pack cost as one of the key limiting factors.  Given the projected decrease in cost of Li-ion batteries, the potential for electrification of mass-market light commercial vehicles is promising.  Alongside, there is increased interest in electrifying freight vehicles and in an earlier viewpoint, we analyzed battery performance requirements for electrifying class-8 semi trucks.  We identified battery specific energy as the major limiting factor which limits the payload carrying capacity and showed that in the near term, electric semi-trucks will be limited to ranges up to 600 miles.  However, for semi-trucks with those ranges, there can be a compelling economic case if \vv{Add the specific targets.}


% Two recent works have set the stage carrying out a first-order performance assessment for electric aircraft of size \vv{...; add a couple of lines about schafer/barrett papers.}\ab{There are a number of publications similar to the schafer/barrett papers, do we want to stick to schafer/barrett, extend this comment a little more, or simply make a more generalized statement?}  In this work, we build on this to specifically carry out a detailed performance assessment for batteries required for electrifying fixed-wing aircraft, integrating flight aerodynamics along with physics-based battery dynamics model.  Using this model, we identify \vv{give a one-line summary at the end of what we find.}  Based on the identified targets, we present an outlook for potential beyond Li-ion batteries to meet  these targets.

\begin{equation}
    M_{TO} = ewf * M_{TO} + S_e * \int P(t) dt  + M_{pax} 
    \label{Eqn:massSum}
\end{equation}

K is calculated based on the parameters in table 1, and others which can be found in the supplementary information. As for the mission parameters, altitude was set to a constant 8000 meters. Range and payload were left as floating variables, as feasibility of a potential electric aircraft is a function of range and payload.
For each representative aircraft, the takeoff mass is divided into three parts: aircraft structure, energy storage, and payload (equation (\ref{Eqn:massSum}).


In our analysis, we will primarily study the feasibility of electric aircraft as a function of empty weight fraction (EWF) and gravimetric specific energy (SE). ($\mathrm{S_{E}}$). EWF is the weight ratio of the aircraft without battery and payload to $\mathrm{M_{TO}}$. We chose to use EWF rather than a comprehensive weight buildup because we are interested not in the particularities of aircraft weight but in the feasibility of the system as a whole. $\mathrm{S_{E}}$ is the energy stored per unit mass of the battery pack. Thus, at a specified EWF and SE, there exists a trade off between the number of passengers that can be carried (PAX) and the the range (dictated by the weight allocated for energy storage). We will use the number of passengers multiplied by the aircraft range (PNMi) as a figure of merit to study system feasibility at each specified EWF and SE. To ensure that the algorithm does not converge to a corner case (e.g. no passengers and extreme long range or high passengers and extreme short range), we subtract a minimum range and passengers from each value in this calculation. The minima are different for each class and are set by studying historical aircraft of each class. To select the optimal combination of PAX and range, we maximized the available seat-nautical-miles (PNMi). This sets the design point for the most viable aircraft. We then repeated this process within reasonable bounds for EWF and SE to analyze aircraft viability as a function of EWF and SE, and ultimately to identify baseline values for a reasonable electric aircraft within each class. 


Let's now look to the practicality and acchievability of the parameters identified here as the requirements for a fully electric aircraft. First, let's think about SE. The widebody aircraft shown require a specific energy of around 1200 Wh/Kg before they are able to achieve the minimum design parameters for a practical aircraft. However, the aircraft at 1200 Wh/Kg would only have a range on the order of 1000 NMi and could only carry around 200 PAX. Currently, wide body aircraft tend to be used on long flights carrying large numbers of passengers. The distribution of flights by wide body aircraft tends to be normal, and they tend toward the higher end of their range. They do not tend to fly many routes as short as 1000 NMi, as they tend to fly the longer intercontinental routes and international flights. Therefore, in practice, 1200 Wh/Kg is not nearly enough to power such an aircraft. \ab{Need to supply figures with this stuff, whether its from my BTS database or someone else's}. In fact, the specific energy for a practical wide body aircraft would need to be much larger than 2000 Wh/Kg, far greater than anything used today. 

Narrow body aircraft present a slightly more compelling case. Figure 3 shows that the SE required to build the least ambitious narrow body aircraft is around 800 Wh/Kg, which compares well with the number cited by barrett et. al. This aircraft, much like the comparable wide body aircraft, would be at the very lowest end of what would be considered practical, and would probably not be economically feasible for many narrow body fleet operators. It would be unable to carry out missions such as transcontinental flights which are currently executed by narrow body aircraft. However, some narrow body flights would be able to be achieved, particularly on short flights between hubs. This transitional class and range of required SE is best illustrated by Figures 11 and 12, which show the range distributions for two narrow body aircraft. Figure 11 shows the distribution of cruise range of the airbus a319 aircraft, and figure 12 shows the airbus a321 range distribution. While both of these aircraft are part of the a320 family of narrow body aircraft, the a319 flys much shorter routes than does the a321, and therefore it will be much easier to electrify that aircraft. On the other hand, the a321 tends to fly much longer routes and will be much later to be electrified.

Regional aircraft present the most compelling case for electrification. Figure 5 shows the range at various SE and EWF for regional aircraft. These aircraft become possible at around 400 Wh/Kg and practical at around 600 Wh/Kg. Moreover, regional aircraft also tend to fly the shortest routes of any of the aforementioned categories. However, regional aircraft will face their own challenges in the course of electrification. It should be noted that the regional aircraft simulated here is heavier than many of the conventional regional aircraft currently in service, which will bring with it its own set of challenges.  

\ab{Section about batteries and capabilities of batteries, use this section to talk about discharge rates}

Other performance metrics also predict that smaller planes will be easier to electrify. EWF for comparable aircraft can be much higher for a regional aircraft than for a wide body or narrow body one. EWF is much more difficult to estimate for an electric aircraft than for a conventional one, in part because of the many faceted nature of this metric. However, many conceptual designs of electric aircraft (Barrett et al, add some more), show an EWF that is lower than than the current capabilities of electric aircraft. In any case, EWF needs to be lower than current capabilities to enable practical electric aircraft. 

No electric aircraft could approach the maximum range of an existing fuel burning aircraft, even considering that the electric aircraft considered here have higher MTOM relative to the fuel burning aircraft (and thus more weight can be budgeted for energy storage). See Figures 8, 9, and 10 for a comparison of some electric aircraft with conventional aircraft in similar classes. Notice especially that in the higher end (longer range and higher PAX), the aircraft are larger than conventional aircraft, require higher specific energies, and lower empty weight fractions. However, statistical analysis \textbf{cite} of actual/real-world commercial flights reveals a (gamma?) distribution heavily weighted to the low end. This means the difference between being able to fly all routes and simply to fly many routes is significant.  






Narrow-bodied aircraft can first transport a reasonable number of passengers \lf{as specified above} at least 500 nmi at a specific energy of \textbf{sumpin} and an empty weight fraction of \textbf{sumpin else}. Relaxing the empty weight constraint to the most optimistic only brings this down to \textbf{more sumpin}, while the current average empty weight fraction for this class of aircraft, \textbf{insert yet anpther sumpin here}, would push the required specific energy up to \textbf{last sumpin}.

From this, it becomes apparent that regional jets offer not only the most achievable application in terms of both battery and aerospace technologies, but also that reaching these advancements would be more viable than the maximum range would suggest, being able to carry out a significant number of the 20\% overall of commercial flights relegated to regional jets.




\lf{the next two paragraphs seem trivial, but will leave to shashank to determine necessary level of explanation}










\textbf{Not sure where to go from here until the battery analysis and ultimate points get nailed down}


\begin{itemize}
    \item Looking at the figure, certain range(s)/points have been identified, which correspond to what we have determined to be the minimum acceptable range/performance based on flight data
    
    \item The three classes of aircraft considered are [@alec fill in] because the occupy [x,y,z] percent of the total number of flights respectively.
        \subitem - does this belong earlier where maybe we explain the different masses we look at, assuming we look at multiple masses
        
    \item While previous investigations have compared the viability of electric aircraft by comparing their achievable range to the maximum achievable range of the comparable classes of jet-fueled aircraft (Barrett, others?), the data suggest that electric aircraft would not require the same capabilities to compete.
    
    \item Analyzing ADBS(or whatever it is) data / using data collected and analyzed by (cite that guy who does this stuff), most/many/all commercial flights follow a gamma distribution (example(s) in SI) with a peak/mean much closer to the low end of ranges traveled than the max available range
    \item add any data analysis that Alec did personally that is interesting and relevant, 2-3 lines probably
    \item This reassigns the target range for viability from (give stuff characterized by barrett and/or others) to (what we say based on distribution)
        \subitem - probably need a better argument here than just "because it can handle the majority of flights it will"
        \subitem - need some reason why airlines would want to adopt electric for the shorter range flights and maintain jet-fuel aircraft for longer ones. My gut instinct is they wouldn't, so what is the financial calculus that gets us there? 
    
    \item Do we want to make a prediction about empty weight here or say something more to the effect of "the issue of achievable weight metrics is contentious(applicable to cite that article that kinda sucked here? prolly no), and is has been estimated at least as low as [blank] in an optimistic case. Therefore, we will operate at that assumption for a representative canonical case, which is applicable if aircraft technology can get there.
    
    \item Similarly, battery estimates have been way too high in past work. Call back to Shashank section if he pointed this out already. We will discuss below (better wording in actual) how we forsee reaching the canonical pack-level energy of [blank], but once again, getting there is likely required for the performance we see as acceptable.
    \item basically specify and defend canonical case
\end{itemize}  
\lf{what are we saying about reserve? Making an assumption I'm sure, but which? talk about power for aborted landing???}
\begin{itemize}
    \item From the target performance identified above, the modeled PNMI is [blank], composed of [blank] passengers traveling a range of/within/atleast [blank] nmi. 
    \item This would cover the range of (say something about the CDF here )
    \item may need to defend the fact that even though we can cover the range, it's at nowhere near the same number of people unless we bump specific energy up to x or drop empty weight to y, so without some major economic advantage it's not likely to be enough JUST to hit the range
    \item autolion? may, actually probably, just be able to say "this is not possible yet, we need next-gen cause Li-ion will never get there"
    \item possible to quantify anything about next-gen, when they could, which ones could handle this profile and specific energy
    \item make points about power if necessary
\end{itemize}

The canonical case we evaluate -- what are the additional requirement that need to understood -- critical areas that aircraft people need to think about\\

\begin{itemize}
    \item discuss figure of canonical case (is this an autolion figure or power profile @shashank and @alec?)
    \item Figure (whatever one has the stuff with the specific energy against the weight ratio) symbolizes the need for integration between battery and aircraft designers. The constraints and optimization of each depend on the implementation of the other.
    \item Of the two principle metrics identified, empty weight fraction and specific energy, the former falls predominantly within the realm of aircraft design.
    \item However, aircraft designers cannot simply improve within their realm with the same process applied to fuel-burning aircraft
    \item One important consideration for aeronautical designers when reviewing battery specifications is the constraint on power density on top of energy density \lf{already discussed?}. Ragone description. 
    \item \lf{this could be a good time to go into reserve? like say that generally in a conventional aircraft this is not a huge problem as long as a fairly full-charged battery can meet climb specs. But then if you have to climb again at the low SOC it can be a problem. Ref the evtol thing?}
    \item I think here would be a good place to point out that in counting on things like Li-S we also need to see an improvement in power density, because if you just scaled up current tech it wouldn't be able to handle even the peak C-rate at climb. \lf{is this a good understanding of the power properties of beyond Li-ion or not, @shashank?} 
    \item \lf{also possible the power density belongs in the next section. It's here as a "make sure you know the specifications you need as an aircraft designer" but maybe it's more of a "hey battery peeps, I know you are used to optimizing for specific energy for the EV market, but you need to also improve specific  power to meet the demands of this particular application"}
    \item Part of integrating the design is realizing you have to do like we are trying to do with airbus, i.e. investigate how a pack degrades with characteristic use and iteratively incorporate the pack sizing to accommodate capacity AND power fade. Battery people need aircraft people because the design and use profile will change as the pack is changed, and aircraft pople need battery people because they cannot simply assume a certain set of power and energy density, nor a simple way they change with time. \lf{this seems like a good candidate for a main point to make with the analysis/battery stuff. If we transition the viewpoint from "this is what is feasible" which has been done a lot and also thouroughly by barrett, what if we start out with our "yes... but" plot and then say "the important thing is that it is not just a case of either side giving a simple value to the other (e.g. power profile or in the reverse specific energy and power), rather it is an iterative process that depends on integration in the initial design stage}
    
\end{itemize}


Critical areas that battery people are most likely to miss -- coupling of the two aero-battery -- what are some new challenges that have not been outlined by others\\
\begin{itemize}
    \item battery charging: something to keep in mind from the battery side is the challenge of refueling/recharging the aircraft. Currently, aircraft refuel as they are onboarding and unloading people, which results in a large percentage of their life in the air. (cite) This is important for maintaining the thin profit margins of commercial airlines. To be competitive -- with the already reduced number of passengers than could be transported a given distance -- an electric aircraft would similarly need to ensure a fast turnover on the ground. This presents 2(+) challenges.
    \item First, fast charging is already a problem in the automotive world with problems like accelerated decay, cell limitations, and even the ability to supply the massive currents required even if it could all be effectively transferred to the battery.
    \item On top of this, safety is a serious concern in the charging stage. Most cases of thermal runaway occur while a cell is charging(source?), and with the higher C rates in general in aircraft relative to electric ground vehicles, the packs are far from pre-cooled heading into charging, assuming there is very little rest between the mission and the beginning of charge. 
    \item \lf{how are the beyond Li-ion cells at safety when charging?}
    \item The equivalent of the current model would have batteries charging while passengers are on the aircraft, which is not likely an acceptable condition unless battery engineers can show complete containment for (the worst case i think) failure. \lf{we can bring in that thing they showed us at airbus about the safety requirements that say more or less this, more or less like shashank did in the group presentation}. 
    \item As a result battery engineers will need to account for very fast charge so that the ground time lost to charging is minimized, or greatly improve on the safety of pack design to meet the targets outlined(?). This will affect the packing burden factor, which will further limit the specific energy attained for flight applications
    \item Talk about stuff that limits the theoretical pack specific energy of beyond Li-ion?
    \item reserve stuff: shift the discussion of the specific power and effect of needing a large power output at the end of discharge to here?
    \item Make point about sizing to the power-fade condition
\end{itemize}  

The estimation of the energy and power requirements is accomplished using a first principles flight dynamics model.  We have used such an approach for analyzing electrification of light commercial vehicles,\cite{sripad2017evaluation} semi-trucks,\cite{sripad2017performance} eVTOLs.\cite{fredericks2018performance}   Four forces act on an aircraft in flight: thrust (force generated by the propulsion system), drag (aerodynamic force opposite to the direction of motion), weight (gravitational force), and lift (aerodynamic force normal to the wings). The power at any point of flight is the product of thrust and velocity, resulting in Eq. \eqref{eqn:masterPower}. The energy is given by the integral of instantaneous power, given in Eq. \eqref{eqn:masterPower}, over the duration of the mission.